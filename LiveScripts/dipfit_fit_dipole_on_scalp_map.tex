% This LaTeX was auto-generated from MATLAB code.
% To make changes, update the MATLAB code and export to LaTeX again.

\documentclass{article}

\usepackage[utf8]{inputenc}
\usepackage[T1]{fontenc}
\usepackage{lmodern}
\usepackage{graphicx}
\usepackage{color}
\usepackage{listings}
\usepackage{hyperref}
\usepackage{amsmath}
\usepackage{amsfonts}
\usepackage{epstopdf}
\usepackage{matlab}

\sloppy
\epstopdfsetup{outdir=./}
\graphicspath{ {./dipfit_fit_dipole_on_scalp_map_images/} }

\begin{document}

\matlabtitle{Using DIPFIT to fit one dipole to EEG or ERP scalp maps}

\vspace{1em}

\begin{par}
\begin{flushleft}
This Live Script shows how to use EEGLAB command-line function to use the DIPFIT plugin to fit dipoles to raw ERP or EEG scalp maps. Note that it can't be done through a time-window, you need to specify a time point.
\end{flushleft}
\end{par}

\vspace{1em}

\matlabheadingtwo{Load data}

\begin{matlabcode}
eeglab; close; % add path
\end{matlabcode}

\begin{matlabcode}
eeglabp = fileparts(which('eeglab.m'));
EEG = pop_loadset(fullfile(eeglabp, 'sample_data', 'eeglab_data_epochs_ica.set'));
\end{matlabcode}


\matlabheadingtwo{Find the 100-ms latency data frame.}

\begin{par}
\begin{flushleft}
 Fitting may only be  performed at selected time points, not throughout a time window. 
\end{flushleft}
\end{par}

\vspace{1em}

\begin{matlabcode}
latency = 0.100;
pt100 = round((latency-EEG.xmin)*EEG.srate);
\end{matlabcode}

\matlabheadingthree{Find the best-fitting dipole for the ERP scalp map at this timepoint}

\begin{par}
\hfill \break
\end{par}

\begin{matlabcode}
erp = mean(EEG.data(:,:,:), 3);
dipfitdefs;
\end{matlabcode}

\matlabheadingtwo{Specify DIPFIT settings using MNI BEM model}

\begin{matlabcode}
EEG = pop_dipfit_settings( EEG, 'hdmfile',template_models(2).hdmfile,'coordformat',template_models(2).coordformat,...
    'mrifile',template_models(2).mrifile,'chanfile',template_models(2).chanfile,...
   'coord_transform',[0.83215 -15.6287 2.4114 0.081214 0.00093739 -1.5732 1.1742 1.0601 1.1485] ,'chansel',[1:32] ); 
\end{matlabcode}

\begin{matlabcode}
[ dipole, model, TMPEEG] = dipfit_erpeeg(erp(:,pt100), EEG.chanlocs, 'settings', EEG.dipfit, 'threshold', 100);
\end{matlabcode}


\matlabheadingtwo{Plot the dipole on 3-D map}

\begin{matlabcode}
pop_dipplot(TMPEEG, 1, 'normlen', 'on');
\end{matlabcode}
\begin{matlaboutput}
No MRI file given as input. Looking up one.
\end{matlaboutput}

\matlabheadingtwo{Plot the dipole and the scalp map}

\begin{matlabcode}
figure; pop_topoplot(TMPEEG,0,1, [ 'ERP 100ms, fit with a single dipole (RV ' num2str(dipole(1).rv*100,2) '%)'], 0, 1);
\end{matlabcode}
\begin{matlaboutput}
Plotting...
\end{matlaboutput}
\begin{center}
\includegraphics[width=\maxwidth{58.50476668339187em}]{figure_0}
\end{center}

\begin{center}
\includegraphics[width=\maxwidth{59.508278976417465em}]{figure_1}
\end{center}


\end{document}
